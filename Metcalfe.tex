\documentclass[12pt, a4page]{article}

%\usepackage[backend=biber, defernumbers=true]{biblatex}
%\addbibresource{biblio.bib}

\title{Metcalfe's Law}
\date{}
\author{Pranav Kasela, Federico Moiraghi}

\begin{document}
\maketitle

\part*{Socio-Economic Interpretation.}
The formula of Metcalfe's Law presupposes that each node is linked with the whole net and all connections have the same importance.
Those presuppositions are considered too strong, so new mathematical laws have been presented; but empirical studies have demonstrated the veracity of Metcalfe's Law in social networks. \newline
Presuppositions of Metcalfe's Law are not satisfied even when its veracity is demonstrated; however, in a social environment other presuppositions are to be considered: user's behaviour and user's \textit{saturation}; moreover, algorithms used to improve links quality and the possibility for people to aggregate may be enough to make the value of the whole network increase according to a quadratic function $n^2$. \newline
In general, all laws that try to estimate the value as a function of the number of nodes do not consider that the network in made by people connected, and so human behaviour must be considered. \newline

Estimating the value of a net is important in an economic contest: costs and earnings are the base of any business. Costs are not considered just in terms of money, but also as resources needed to run the business, like server's disk space and computational power.
Decisions concerning the server depends on the costs of the network; and since it is necessary to run the business, estimation can not be lower than real costs: the service risks to be interrupted and the value of the network drastically decreases to zero. \newline

\section{Economic Evaluation of a Net.}
In general, to a company, nets are not the product but just a distribution channel: in this case, the net can be evaluated according to earnings coming from it; in this case the model is just linear.
This is just a simple and empirical way to estimate a value, but works for big companies that offer a payment service. \newline
A generalization of this criterion is to calculate the sum of value exchanged between each nodes; this can be easily done in e-commerce sites or for a transport net. \newline%\footfullcite{Weinman}.

Metcalfe's Law is necessary when no or small data are available, so the value can not be known and can be just estimate.
This is the case of advertisement selling: the value of the space is not known, but companies have to accord a price according to the distribution that the advert may have in the network. \newline
 
Estimating the value is important when balance-sheet is compiled or budget is calculated. In addiction, according to Italian's Law, the balance-sheet must be \textit{prudent}, so costs can not be under-estimated.
With this in mind, Metcalfe's Law is correctly adopted: if costs follow a $o(n^2)$ model, laws would not be not broken; in fact, it is empirically demonstrated that costs of a social media follows this law.
Using the same criterion, if no data are available, earnings are not well estimated: is more prudent using a lower formula, like $n \cdot \log{n}$. \newline

\section{Net saturation.}
In a network, often nodes are normal people: in a finished time, not all value can be extracted.
This means that a person has no time to communicate with and may not receives all communications from the whole net.
The term \textit{saturation} is used to mean that a person can not exchange the maximum value with the net due to the mole of information that receives.
In addition, not all communications have the same value: user is covered with information, for the most useless, that can not manage.


%\newpage
%\printbibliography[type=article]

\end{document}
