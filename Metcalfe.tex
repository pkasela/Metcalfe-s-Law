\documentclass[12pt, a4page]{article}

\usepackage[backend=biber]{biblatex}
\addbibresource{biblio.bib}

\title{Metcalfe's Law}
\date{}
\author{Pranav Kasela, Federico Moiraghi}

\begin{document}
\maketitle

\part*{Socio-Economic Interpretation.}
Metcalfe's Law assumes that each node is linked with the whole net and all connections have the same importance.
Those assumptions are considered too strong, so new mathematical laws have been presented; but empirical studies have demonstrated the veridicity of Metcalfe's Law in popular social networks\cite{Metcalfe}\cite{Wechat} even if premises of are not satisfied.
However, in a social environment the human nature of nodes must be considered; morover, algorithms used to improve connections quality and the possibility given by the platform to \textit{share} contents may be enought to make the value of the whole network increase according to a quadratic function $n^2$. \newline

Extimating the value of a net is important in an economic contest: costs and earnings are the base of any business. Costs are not considered just in terms of money, but also as resources needed to run the business, like server's disk space and computational power.
Decisions concerning the server depends on the costs of the whole network; and since it is necessary to run the business, extimation can not be lower than real costs: the service risks to be interrupted and the value of the network drastically decreases to zero. \newline

\section{A real case: Facebook.}
Facebook, the biggest network in the world owned by a public company, is demonstrated to have a value proportional to the square of active users\cite{Metcalfe}\cite{Wechat}.
But assumptions of Metcalfe's Law are apparently not satisfied due to the limit to the number of \textit{friends} that a single person can have and the fact that users do select their contacts according to various strategies\cite{Socialmedia}. \newline

However, this is not a problem, because Facebook's algorithm and the possibility to \textit{share} contents make links between people that apparently are not connected: the exchange of value is imposed by Facebook, even without user's acception.
According to Pasquali\cite{Socialmedia}, Facebook is not a single network, but an aggregate of small (social) networks that exist also offline and a big (informatic) net composed by the aggregate of all users.
While small social networks are not intresting in this study, the fact that users can interact with the whole Facebook is foundamental.
Users, in fact, can \textit{share} contents, so that they can reach more nodes (\textit{friends of friends} or \textit{public pages}), or, in particular cases, a \textit{post} can become \textit{viral}.
In addiction to this, Facebook's algorithm minds the creation of new links, suggesting new contents that is sure to be liked by the user or suggesting new people or pages.

The value exchanged through Facebook's network is maximized by the algorithm: not all informations are presented to the user, but they are filtered according to personal interests.
The unique experience during the \textit{flow} of feeds can cause dependency: the time spent on the platform, the facility of access and the quality (or targeting) of contents are so hight that saturation is not easily reach. \newline

\section{Economic Evaluation of a Net.}
In general, networks are not the product of the business but just the distribution channel: in this case, the net can be evaluated equal to earnings coming from it; in this case the model is just linear.
This is just a simple and empirical way to extimate a value, but works even for big companies that offer a payment service. \newline
A generalisation of this criterion is to calculate the sum of value exchanged between each node; this can be easily done in e-commerce sites or for a transport nets\cite{Weinman}
. \newline

Metcalfe's Law is necessary when no or small data are avaible, so the value can not be known and can be just extimate.
This is the case of advertisment selling: the value of the space is not known, but companies have to define a price according to the distribution that the advert may have in the network. \newline

Extimating the value is important when balance-sheet is compiled or budget is calculated. In addiction, according to Italian's Law, the balance-sheet must be \textit{prudent}, so costs can not be under-extimated.
With this in mind, Metcalfe's Law is correctly adopted: if costs follow a $o(n^2)$ model, law would not be broken; in fact, it is empirically demonstrated\cite{Wechat} that costs of a social media follows this law.
Using the same criterion, if no data are avaible, earnings are not well extimated: is more prudent using a lower formula, like $n \cdot \log{n}$. \newline

\section{Net saturation.}
In a network, nodes are often normal people: in a finished time not all the value transmitted can be extracted.
This means that a person has no time to communicate with and may not receives all communications from the whole network.
The term \textit{saturation} is used to mean that a person can not exchange the maximum value with all other nodes due to the mole of informations that receives.
In addition, not all communications have the same value: user is covered with informations, for the most unusefull, that can not manage. \newline
To maximize the value that a network can transfer, filters of any type are used: from simple spam filters, to automatize the check of emails, to server algorithms that suggest new actions and connections.
In this way, the value of links increases due to the growth of information that the user can manage. \newline

\section{Conclusions.}
The value of a network is not easily extimable: in general, the value is inferior to $n^2$ presented by Metcalfe, but with some tricks is possible to a net to reach the maximum value.
In particular, Facebook's network, due to its possibility to \textit{share} contents and a sophisticated algorithm that uses tons of informations about each user, can be evaluated according to Metcalfe's Law; but this mathematical function is not always valid, even in social environment, due to human nature: social networks are not so inter-connected, and a informatic infrastructure mirrors the offline world. 
And more in general, other laws, of minor magnitude, are empirically and suxcessfully used.
\newpage
\printbibliography

\end{document}