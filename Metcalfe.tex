\documentclass[12pt, a4page]{article}

%\usepackage[backend=biber, defernumbers=true]{biblatex}
%\addbibresource{biblio.bib}

\title{Metcalfe's Law}
\date{}
\author{Pranav Kasela, Federico Moiraghi}

\begin{document}
\maketitle

\part*{Socio-Economic Interpretation.}
Metcalfe's Law assumes that each node is linked with the whole net and all connections have the same importance.
Those assumptions are considered too strong, so new mathematical laws have been presented; but empirical studies have demonstrated the veridicity of Metcalfe's Law in social networks. \newline %\footfullcite{Metcalfe}\footfullcite{Wechat}
Premises of Metcalfe's Law are not satisfied even when its veridicity is demonstrated; however, in a social environment user's behaviour and user's \textit{saturation} must be considered; morover, algorithms used to improve connection quality and the possibility for people to aggregate may be enought to make the value of the whole network increase according to a quadratic function $n^2$. \newline
In general, all laws that try to extimate the value as a function of the number of nodes do not consider that the network is composed by people, and so human behaviour must be considered. \newline

Extimating the value of a net is important in an economic contest: costs and earnings are the base of any business. Costs are not considered just in terms of money, but also as resources needed to run the business, like server's disk space and computational power.
Decisions concerning the server depends on the costs of the whole network; and since it is necessary to run the business, extimation can not be lower than real costs: the service risks to be interrupted and the value of the network drastically decreases to zero. \newline

\section{Economic Evaluation of a Net.}
In general, to a company networks are not the product but just the distribution channel: in this case, the net can be evaluated according to earnings coming from it; in this case the model is just linear.
This is just a simple and empirical way to extimate a value, but works even for big companies that offer a payment service. \newline
A generalisation of this criterion is to calculate the sum of value exchanged between each node; this can be easily done in e-commerce sites or for a transport net. \newline%\footfullcite{Weinman}. \newline

Metcalfe's Law is necessary when no or small data are avaible, so the value can not be known and can be just extimate.
This is the case of advertisment selling: the value of the space is not known, but companies have to define a price according to the distribution that the advert may have in the network. \newline
 
Extimating the value is important when balance-sheet is compiled or budget is calculated. In addiction, according to Italian's Law, the balance-sheet must be \textit{prudent}, so costs can not be under-extimated.
With this in mind, Metcalfe's Law is correctly adopted: if costs follow a $o(n^2)$ model, law would not be broken; in fact, it is empirically demonstrated%\footfullcite{Wechat}
 that costs of a social media follows this law.
Using the same criterion, if no data are avaible, earnings are not well extimated: is more prudent using a lower formula, like $n \cdot \log{n}$. \newline

\section{Net saturation.}
In a network, nodes are often normal people: in a finished time not all the value transmitted can be extracted.
This means that a person has no time to communicate with and may not receives all communications from the whole network.
The term \textit{saturation} is used to mean that a person can not exchange the maximum value with all other nodes due to the mole of informations that receives.
In addition, not all communications have the same value: user is covered with informations, for the most unusefull, that can not manage. \newline
To maximize the value that a network can transfer, filters of any type are used: from simple spam filters, to automatize the check of emails, to server algorithms that suggest new actions and links.
In this way, the value of connections increases due to the growth of information that the user can manage. \newline

\section{A real case: Facebook.}
Facebook, the biggest network in the world owned by a public company, is demonstrated to have a value proportional to the square of active users. %\footfullcite{Metcalfe}\footfullcite{Wechat}
Assumptions of Metcalfe's Law are apparently not satisfied due to the limit to the number of \textit{friends} that a single person can have and the fact that people, empirically, do not communicate with the whole mankind.
This is not a problem, because user's behaviour and facebook algorithm make links between people that apparently are not connected.
According to Pasquali %\textcite{Pasquali}
, Facebook is not a single network, but an aggregate of small (social) networks that exist also offline and a big net composed by the aggregate of all users.
This means that, due to \textit{public pages}, more people are connected; and Facebook's algorithm minds the creation of new links suggesting new contents to the user according to his/her own interests. \newline
But all of this is not enought to link all nodes together: it is necessary to consider the possibility to \textit{share} a content, so that it can reach all nodes (or at least who can take value from it), becoming \textit{viral}. \newline

People's behaviour is changed by the use of Facebook and the popularity of smartphones: the meaning of a \textit{post} is different, becoming no more a souvenir or a link to people far away, but an integration of the present; while the costant presence of a portable access point (the smartphone) to the network facilitate communications. \newline
The value exchanged through Facebook's network is maximized by the algorithm: not all contents are presented to the user, but they are filtered according to personal interests.
The unique experience during the \textit{flow} of feeds can cause dependency: the time spent on the platform, the facility of access and the quality (or targeting) of contents are so hight that saturation is not reach. \newline

All this peculiarity of the network and the new behaviour of the user can explain why the value of Facebook grows according to Metcalfe's Law.


%\newpage
%\printbibliography[type=article]

\end{document}
